%%%%%%%%%%%%%%%%%%%%%%%%%%%%%%%%%%%%%%%%%%%%%%%%%%%%%%%%%%%%%%%%%%%%%%%%%%%%%
%	e-Yantra, IIT-Bombay

%	Document Author: Saurav Shandilya
%	Date: 16-August,2012
%	Last Editted by: Saurav
%   Date Last updated: 31-05-2016 

%%%%%%%%%%%%%%%%%%%%%%%%%%%%%%%%%%%%%%%%%%%%%%%%%%%%%%%%%%%%%%%%%%%%%%%%%%%%%

\documentclass[11pt,a4paper]{article}
\usepackage{graphicx}
\usepackage{hyperref}
\usepackage{graphicx}
\title{Drone -Test Flight 2}
\author{keyur Rakholiya}
\author{Akshit Gandhi}
\date{\today}

\begin{document}
	\maketitle
	\newpage
	\tableofcontents
	\newpage
	\section{Tutorial Name}
	Test Flight 2
	


	


	\section{Theory and Description}
		when we went for test flight 2. everything was alright. we had checked all the soft parameters of the Drone. we have also took care about atmosphere. but sometimes very small mistakes leads to crash. we have not given too much attention to the hardware. so during the flight at the height of 6 meter ,oe of the propeller slipped out. and flight got crashed.
		as you can see in video...
		
		\url{https://youtu.be/dcHOmJOeLvU}
		\newline
		
	\section{References}
		\paragraph\url{https://github.com/eYSIP-2016/Autonomous-Drone}
	
\end{document}



