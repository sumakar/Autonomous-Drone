%%%%%%%%%%%%%%%%%%%%%%%%%%%%%%%%%%%%%%%%%%%%%%%%%%%%%%%%%%%%%%%%%%%%%%%%%%%%%
%	e-Yantra, IIT-Bombay

%	Document Author: keyur rakholiya and akshit gandhi
%	Date: 16-August,2012
%	Last Editted by: Saurav
%   Date Last updated: 31-05-2016 

%%%%%%%%%%%%%%%%%%%%%%%%%%%%%%%%%%%%%%%%%%%%%%%%%%%%%%%%%%%%%%%%%%%%%%%%%%%%%

\documentclass[11pt,a4paper]{article}
\usepackage{graphicx}
\usepackage{hyperref}
\usepackage{graphicx}
\title{AUTO TAKEOFF AND LANDING}
\author{keyur Rakholiya}
\date{\today}

\begin{document}
	\maketitle
	\newpage
	\tableofcontents
	\newpage
	\section{Tutorial Name}
	\hspace{0.5in}AUTO TAKEOFF AND AUTO LANDING
	
	


	


	\section{Theory and Description}
		we have connected R pi and Apm through UART. and we have written a code in python. so after running this code on r pi.drone will take off autonomously without any control. it will go up to hight of 2 m. then after 10 seconds it will go in LAND mode.\\
		you can see video on youtube. \\
		\\
		\url{https://www.youtube.com/watch?v=GHcQYoiWwgk}\\
		\\
		\\
		code is here:\\
		\\
		https://github.com/eYSIP-2016/Autonomous-Drone
		\newline
		

	
\end{document}



